\section{k-Server-Problem}

\begin{takeaway}
    \item k-Server-Problem
\end{takeaway}

\paragraph{Motivation}
Bewege Objekte in einem Raum zu bestimmten Punkten.
Z.B. Polizisten von Dienststellen zu crime scenes.

\paragraph{Metrischer Raum}
Sei $S$ eine Menge von Punkten, sei $\dist : S \times S \mapsto \mathbb{R}$ eine Distanzfunktion.
$\mathcal{M}(S, \dist)$ ist ein \emph{metrischer Raum} falls gilt:
Definitheit, Symmetrie, Dreiecksungleichung.

Beispiel: Euklidischer Raum. Vollständige, gewichtete, ungerichtete Graphen mit Dreiecksungleichung.
\\
Beobachtung: Alle Graphen mit Kastenkosten $\in \{1, 2\}$ erfüllen die Dreiecksungleichung.

\paragraph{k-Server}
Sei $\mathcal{M}(S, \dist)$ ein metrischer Raum.
Sei $s_1, ..., s_k$ Server als Punkte in $S$.
Sei eine Multimenge $C_i \subseteq S$ mit $|C_i|=k$ eine \emph{Konfiguration} von Servern in Zeitschritt $i$.
\\
Die \emph{Distanz} \footnote{Achtung Verwechslungsgefahr!}
zwischen $C_r$ und $C_t$ sind die Kosten eines minimalen Matchings zwischen ihnen.

Eine Instanz $I = (x_1, ..., x_n)$ fragt Punkte an, so dass in Zeitschritt $i$ ein Server nach $x_i$
bewegt werden muss (falls dort noch keiner steht).

Ziel: $\min \sum_i costMinMatching(C_i, C_{i+1})$

\paragraph{Träge}
Ein Online-Algorithmus für k-Server heisst \emph{träge} wenn er nur dann einen Server bewegt,
wenn auf $x_i$ noch kein Server steht.
Auch bewegt er pro Zeitschritt maximal einen Server.

Dies erleichtert die Analyse. Gleichzeitig gilt (Satz): \\
Jeder c-kompetitive OA für k-Server kann in einen trägen OA umgewandelt werden der auch c-kompetitiv ist.
