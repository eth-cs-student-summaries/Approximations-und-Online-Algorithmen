\section{Online-Rucksackproblem}

\begin{takeaway}
    \item Online-Rucksackproblem
    \item Ansätze: deterministisch, mit Advice, randomisiert
    \item Untere Schranken für den (erwarteten) kompetitiven Faktor und die Advice-Komplexität
\end{takeaway}

\paragraph{Online-Rucksackproblem}
Eingabe $I=(w_1, ..., w_n)$, wobei wir vereinfachen mit \mbox{Gewicht = Wert}.
$w_i \in \R, 0 \leq w_i \leq 1$ (nicht $w_i \in \N$ wie bei Offline!).
In jedem Zeitschritt $i$ muss $A$ entscheiden ob er $w_i$ einpacken will.
Dies kann nicht aufgeschoben oder revidiert werden.

Zulässige Lösung: Menge von Indizes $S \subseteq \{1, ..., n\}$ so dass
$gain(A(I)) := \sum_{i \in S} w_i \leq 1$.
\\
Ziel: $gain(A(I))$ maximieren.


\subsection{Deterministisch}

\paragraph{Satz}
Jeder deterministische OA für das Rucksackproblem hat einen beliebig grossen (d.h schlechten)
kompetitiven Faktor.

\underline{Beweis:}
Schritt 1: Gegenspieler bietet Objekt $w$ mit Gewicht $\varepsilon > 0$ an.
Schritt 2: Falls $w$ akzeptiert, biete Objekt mit Gewicht 1 an. Falls $w$ verworfen, breche ab.
\\
$\implies$ kompetitiver Faktor von $\frac{1}{\varepsilon}$ bzw. von
$\frac{\varepsilon}{0}$/unendlich/undefiniert.

\vspace{5mm}

Trotzdem ist $Greedy$ auf bestimmte Klassen von Eingaben gut:

\paragraph{Satz}
Für jede Instanz mit Gewichten $\leq \beta$ ist $Greedy$ optimal oder erreicht einen Gewinn von $> 1-\beta$.

\underline{Beweis:}
Fallunterscheidung: Gesamtgewicht aller angebotenen Objekte $\leq 1$ (optimal), oder $> 1$ (dann
ist in $Greedy$'s Sack nur $< \beta$ Platz leer).

\paragraph{Satz}
Für jede Instanz mit $gain(Opt(I)) \leq \frac{1}{2}$ ist $Greedy$ optimal.

\underline{Beweis:}
Offensichtlich gilt $gain(Greedy(I)) \leq \frac{1}{2}$.
Falls $gain(Greedy(I)) < gain(Opt(I))$, dann ...
TODO
% Laut Skript: ... muss $Greedy$ ein Objekt mit Gewicht $> \frac{1}{2}$ verworfen haben.
% Mir ist nicht klar ob/warum dies gilt.
% Ich würde argumentieren es kann keine weiteren Objekte geben.
% Denn die mit >=1/2 hätte Opt direkt eingepackt,
% und die mit >1/2 hätte Opt auch eingepackt (und dafür alle eben noch eingepackten verworfen).


\subsection{Mit Advice}

\paragraph{Satz (Triviale untere Schranke)}
Es existiert ein optimaler OA mit Advice für das Rucksackproblem der $n$ Advice-Bits benutzt.

\underline{Beweis}: Left as an exercise to the reader.

\paragraph{Satz (Scharfe Schranke)}
Jeder optimale OA mit Advice für das Rucksackproblem muss mindestens $n-1$ Advice-Bits benutzen.

\underline{Beweis:}
Konstruiere Klasse von Instanzen:
$ I = (\frac{1}{2}, \frac{1}{4}, ... , \frac{1}{2^{n-1}}, w_b)$
wobei $ w_b = 1 - \sum_{i=1}^{n-1} b_i 2^{-i} $ für einen beliebigen Binärstring $b$ der Länge $n-1$.
Für $n=8$ z.B. führt $b=1101101$ zu
$$ w_b = 1 - \left( \frac{1}{2} + \frac{1}{4} + 0 + \frac{1}{16} + \frac{1}{32} + 0 + \frac{1}{128} \right) $$
Die eindeutige optimale Lösung hat Gewinn 1, und füllt mit den ersten $n-1$ Objekten die ``Lücken''
von $w_b$ auf.

Es gibt $2^{n-1}$ veschiedene Binärstrings der Länge $n-1$, d.h. es gibt ebenso viele Instanzen.
Ein optimaler OA mit Advice muss alle unterscheiden können, d.h. er braucht $n-1$ Advice-Bits.
\\
Mit weniger Bits verhält er sich auf zwei Instanzen gleich (Schubfachprinzip),
und kann nicht auf beide optimal sein.

\paragraph{Algorithmus: KPone}
Lese ein Advice-Bit. Es entscheidet zwischen: Greedy von Beginn an, oder Warten auf ein Objekt
mit Gewicht $> \frac{1}{2}$ und ab dann Greedy.

\paragraph{Satz}
Der OA mit Advice $KPone$ für das Rucksackproblem ist strikt 2-kompetitiv.

\underline{Beweis:}
\\
Fall 1: ein Objekt mit Gewicht $> \frac{1}{2}$ existiert. $KPone$ packt dieses ein,
$\implies gain(KPone(I)) > \frac{1}{2}$.
\\
Fall 2: Setze $\beta = 2$ und wende obigen Satz für det. OAs an
$\implies gain(KPone(I)) \geq 1 - \frac{1}{2} = \frac{1}{2}$ (oder optimal).

Dann gilt:
$$ \frac{gain(Opt(I))}{gain(KPone(I))} \leq \frac{1}{gain(KPone(I))} \leq \frac{1}{\frac{1}{2}} = 2 $$

\paragraph{Fazit}
Mit $n-1$ Advice-Bits sind wir optimal, mit 1 Advice-Bit bereits 2-kompetitiv.
Konstant mehr Bits helfen nicht.

Untere Schranke: kein OA mit $< \log_2 (n-1)$ Advice-Bits kann besser als $(2-\varepsilon)$-kompetitiv sein.
\footnote{Beweis ähnlich wie oben, via Konstruktion einer Klasse von Instanzen und Schubfachprinzip.}
\\
Obere Schranke: mit $\bigO(\log n)$ Advice-Bits kommen wir beliebig nah an die optimale Lösung heran.


\subsection{Randomisiert}

\paragraph{Motivation}
Ein Advice-Bit ist mächtig. Wie mächtig ist ein Zufallsbit?

\paragraph{Algorithmus: RKPone'}
Wie KPone, aber statt Advice-Bit nimmt er ein Zufallsbit.

\paragraph{Satz}
Der Barely-Random-Algorithmus $RKPone'$ ist strikt 4-kompetitiv im Erwartungswert.
Diese Schranke ist dicht (d.h. er kann nicht besser sein).

\paragraph{Algorithmus: RKPone}
Wähle u.a.r. einen deterministischen OA aus $strat(RKPone) = \{ Greedy_1, Greedy_2 \}$.
\\
$Greedy_1$ ist eine normale Greedy-Strategie.
\\
$Greedy_2$ simuliert $Greedy_1$, akzeptiert aber nichts. Sobald ein Objekt nicht mehr in $Greedy_1$s
Rucksack passt, akzeptiert $Greedy_2$ dieses und folgt von dann an einer Greedy-Strategie.

\paragraph{Satz}
Der Barely-Random-Algorithmus $RKPone$ ist strikt 2-kompetitiv im Erwartungswert.

\underline{Beweis:}

Fall 1: $\sum w_i \leq 1$ $\implies$ $gain(Greedy_1(I)) = gain(Opt(I)) \leq 1$ und $gain(Greedy_2(I))=0$ \\
$\implies$ erwartete Gewinn = $\frac{1}{2} \cdot gain(Opt(I))$

Fall 2: $\sum w_i \geq 1$. Dann ist der erwartete Gewinn:
$$ \frac{1}{2} gain(Greedy_1(I)) + \frac{1}{2} gain(Greedy_2(I))
= \frac{1}{2} \left( gain(Greedy_1(I)) + gain(Greedy_2(I)) \right) \geq \frac{1}{2} $$
wobei wir verwenden dass $gain(Greedy_1(I)) + gain(Greedy_2(I)) \geq 1$
(da $Greedy_2$ mindestens das Objekt akzeptiert mit dem $Greedy_1$ die Rucksackkapazität überschritten hätte).

\paragraph{Satz (Untere Schranke)}
Kein randomisierter OA für das Rucksackproblem ist besser als strikt 2-kompetitiv im Erwartungswert.

Intuitiv: ein Advice-Bit ist genauso mächtig wie beliebig viel Randomisierung.

\underline{Beweis:}
Betrachte die Klasse von Instanzen $\mathcal{I} = \{ I_1=(\varepsilon ), I_2=(\varepsilon , 1) \} $.
Sei $Rand$ ein randomisierter OA der $x_1$ mit Wahrscheinlichkeit $p$ akzeptiert.
\footnote{$p>0$ muss gelten da $Rand$ sonst auf $I_1$ einen erwarteten Gewinn von 0 hat.}

Für den kompetitive Faktor gilt:
$$
KF(I_1) = \frac{\varepsilon}{p \cdot \varepsilon + (1-p) \cdot 0} = \frac{1}{p} \qquad ; \qquad
KF(I_2) = \frac{1}{p \cdot \varepsilon + (1-p) \cdot 1} $$
Durch Gleichsetzen folgt $ p = \frac{1}{2 - \varepsilon} $, und durch Rückeinsetzen in $KF$ folgt
dass der kompetitive Faktor $\geq 2 - \varepsilon$ ist.
