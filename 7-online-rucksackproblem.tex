\section{Online-Rucksackproblem}

\begin{takeaway}
    \item Online-Rucksackproblem
    \item Ansatz: deterministisch, mit Advice, randomisiert
\end{takeaway}

\paragraph{Online-Rucksackproblem}
Eingabe $I=(w_1, ..., w_n)$, wobei wir vereinfachen mit \mbox{Gewicht = Wert}.
$w_i \in \R, 0 \leq w_i \leq 1$ (nicht $w_i \in \N$ wie bei Offline!).
In jedem Zeitschritt $i$ muss $A$ entscheiden ob er $w_i$ einpacken will.
Dies kann nicht aufgeschoben oder revidiert werden.

Zulässige Lösung: Menge von Indizes $S \subseteq \{1, ..., n\}$ so dass
$gain(A(I)) := \sum_{i \in S} w_i \leq 1$.
\\
Ziel: $gain(A(I))$ maximieren.


\subsection{Deterministisch}

\paragraph{Satz}
Jeder deterministische OA für das Rucksackproblem hat einen beliebig grossen (d.h schlechten)
kompetitiven Faktor.

\underline{Beweis:}
Schritt 1: Gegenspieler bietet Objekt $w$ mit Gewicht $\varepsilon > 0$ an.
Schritt 2: Falls $w$ akzeptiert, biete Objekt mit Gewicht 1 an. Falls $w$ verworfen, breche ab.
\\
$\implies$ kompetitiver Faktor von $\frac{1}{\varepsilon}$ bzw. von
$\frac{\varepsilon}{0}$/unendlich/undefiniert.

\vspace{5mm}

Trotzdem ist $Greedy$ auf bestimmte Klassen von Eingaben gut:

\paragraph{Satz}
Für jede Instanz mit Gewichten $\leq \beta$ ist $Greedy$ optimal oder erreicht einen Gewinn von $> 1-\beta$.

\underline{Beweis:}
Fallunterscheidung: Gesamtgewicht aller angebotenen Objekte $\leq 1$ (optimal), oder $> 1$ (dann
ist in $Greedy$'s Sack nur $< \beta$ Platz leer).

\paragraph{Satz}
Für jede Instanz für die $gain(Opt(I)) \leq \frac{1}{2}$ ist, ist $Greedy$ optimal.

\underline{Beweis:}
Offensichtlich gilt $gain(Greedy(I)) \leq \frac{1}{2}$.
Falls $gain(Greedy(I)) < gain(Opt(I))$, dann ...
TODO
% Laut Skript: ... muss $Greedy$ ein Objekt mit Gewicht $> \frac{1}{2}$ verworfen haben.
% Mir ist nicht klar ob/warum dies gilt.
% Ich würde argumentieren es kann keine weiteren Objekte geben.
% Denn die mit >=1/2 hätte Opt direkt eingepackt,
% und die mit >1/2 hätte Opt auch eingepackt (und dafür alle eben noch eingepackten verworfen).


\subsection{Mit Advice}

\paragraph{Satz (Triviale untere Schranke)}
Es existiert ein optimaler OA mit Advice für das Rucksackproblem der $n$ Advice-Bits benutzt.

\underline{Beweis}: Left as an exercise to the reader.

\paragraph{Satz (Scharfe Schranke)}
Jeder optimale OA mit Advice für das Rucksackproblem muss mindestens $n-1$ Advice-Bits benutzen.

\underline{Beweis:}
Konstruiere Klasse von Instanzen:
$ I = (\frac{1}{2}, \frac{1}{4}, ... , \frac{1}{2^{n-1}}, w_b)$
wobei $ w_b = 1 - \sum_{i=1}^{n-1} b_i 2^{-i} $ für einen beliebigen Binärstring $b$ der Länge $n-1$.
Für $n=8$ z.B. führt $b=1101101$ zu
$$ w_b = 1 - \left( \frac{1}{2} + \frac{1}{4} + 0 + \frac{1}{16} + \frac{1}{32} + 0 + \frac{1}{128} \right) $$
Die eindeutige optimale Lösung hat Gewinn 1, und füllt mit den ersten $n-1$ Objekten die ``Lücken''
von $w_b$ auf.

Es gibt $2^{n-1}$ veschiedene Binärstrings der Länge $n-1$, d.h. es gibt ebenso viele Instanzen.
Ein optimaler OA mit Advice muss alle unterscheiden können, d.h. er braucht $n-1$ Advice-Bits.
\\
Mit weniger Bits verhält er sich auf zwei Instanzen gleich (Schubfachprinzip),
und kann nicht auf beide optimal sein.

\paragraph{Algorithmus: KPone}
Lese ein Advice-Bit. Es entscheidet zwischen: Greedy von Beginn an, oder Warten auf ein Objekt
mit Gewicht $> \frac{1}{2}$ und ab dann Greedy.

\paragraph{Satz}
Der OA mit Advice $KPone$ für das Rucksackproblem ist strikt 2-kompetitiv.

\underline{Beweis:}
\\
Fall 1: ein Objekt mit Gewicht $> \frac{1}{2}$ existiert. $KPone$ packt dieses ein,
$\implies gain(KPone(I)) > \frac{1}{2}$.
\\
Fall 2: Setze $\beta = 2$ und wende obigen Satz für det. OAs an
$\implies gain(KPone(I)) \geq 1 - \frac{1}{2} = \frac{1}{2}$ (oder optimal).

Dann gilt:
$$ \frac{gain(Opt(I))}{gain(KPone(I))} \leq \frac{1}{gain(KPone(I))} \leq \frac{1}{\frac{1}{2}} = 2 $$



\subsection{Randomisiert}

