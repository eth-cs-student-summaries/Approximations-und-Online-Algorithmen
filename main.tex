\documentclass[paper=a4, parskip=half-]{scrartcl}
\usepackage[utf8]{inputenc}

\usepackage{amsmath}
\usepackage{mathtools}
\usepackage{amssymb} % more icons/symbols
\usepackage{hyperref} % for hyper links
\usepackage{graphicx}
\usepackage{enumitem} % for enumerations a), i)
\usepackage{tcolorbox} % for colorful boxes

\usepackage{geometry}
 \geometry{
 a4paper,
 left=20mm,
 right=20mm,
 top=20mm,
 bottom=30mm,
 }


\title{Approximations- und Online-Algorithmen}
\author{\texttt{thgoebel@ethz.ch}}
\date{ETH Zürich, FS 2022}

% Custom commands
\newcommand{\setzeroone}{\lbrace 0, 1 \rbrace} % => {0,1} (Notice the missing $$)
\newcommand{\binarystring}{\setzeroone^*}
\newcommand{\horizontaldivider}{\begin{center} \line(1,0){350} \end{center}}
\newcommand{\A}{\mathcal{A}}
\newcommand{\B}{\mathcal{B}}
\newcommand{\M}{\mathcal{M}}
\newcommand{\N}{\mathbb{N}}
\newcommand{\bigO}{\mathcal{O}}
\newcommand{\bigOstar}{\mathcal{O}^*}
\newcommand{\cost}{\text{cost}}
\newcommand{\Time}{\text{time}}
\newcommand{\st}{\; | \;} % st = such that
\newcommand{\cl}{\; : \;} % cl = colon

% Defines the box to contain the main takeaways from a section that you should be able to explain
\newenvironment{takeaway}{
    \begin{tcolorbox}[colback=red!5!white, colframe=red!65!white, title=Konzepte]
    \begin{itemize}[leftmargin=1em]
    \setlength{\itemsep}{-0.2em}
}
{
    \end{itemize}
    \end{tcolorbox}
}


\begin{document}

\begin{titlepage}
\maketitle
\vspace{5cm}
\thispagestyle{empty}


\begin{abstract}
This documents is a \textbf{short} summary for the course
\textit{Approximations- und Online-Algorithmen} at ETH Zurich.
It is intended as a document for quick lookup, e.g. during revision,
and as such does not replace attending the lecture, reading the slides or reading a proper book.

We do not guarantee correctness or completeness, nor is this document endorsed by the lecturers.
Feel free to point out any errata, either by mail or on
\href{https://github.com/eth-cs-student-summaries/Approximations-und-Online-Algorithmen}{Github}.
\end{abstract}

\end{titlepage}

\tableofcontents
\newpage
%\listoffigures
%\listoftables

%Credits: images are generally taken from the lecture slides.
\newpage

\part{Approximations-Algorithmen}

\section{Approximations-Algorithmen}

TODO.
Siehe das \href{https://github.com/eth-cs-student-summaries/Algorithmik-fuer-Schwere-Probleme/}{Skript von letzem Jahr}.

\newpage

\section{Approximationsschemata}

\begin{takeaway}
    \item PTAS, FPTAS
    \item SKP: 2-Approximation (Greedy), PTAS-SKP
    \item KP: DPKP, FPTAS
\end{takeaway}

\paragraph{Definition PTAS und FPTAS}
Eingabe $(I, \varepsilon)$.
PTAS: Laufzeit ist polynomiell in $|I|$ und beliebig in $\varepsilon^{-1}$.
FPTAS: Laufzeit ist polynomiell in $|I|$ \underline{und} in $\varepsilon^{-1}$.
Approximationsgüte $(1+\varepsilon)$.
Siehe [ASP].

\paragraph{Einfaches Rucksackproblem (Simple Knapsack Problem SKP)}
Gewichte = Kosten. NP-schwer.
Siehe [ASP].

\paragraph{Greedy-SKP}
2-Approximation. Absteigend sortieren, $\bigO(n \log n)$.
Siehe [ASP].

\paragraph{PTAS-SKP}
$(1+\varepsilon)$-Approximation.
$k \gets \lceil \frac{1}{\varepsilon} \rceil$.
Optimale Lösung für alle $\bigO(n^k)$ Teilmengen der Grösse $k$. Dann greedy erweitern in je $\bigO(n)$.
Siehe [ASP].

\paragraph{Allgemeines Rucksackproblem (KP)}
Eingabe $I = (w_1, ..., w_n, c_1, ..., c_n, b)$.
Siehe [ASP].

\paragraph{Exakter Algorithmus für KP (DPKP)}
Dynamische Programmierung. Siehe [ASP].

Sei $I_i$ die Teilinstanz der ersten $i$ Elemente.
Berechne Tripel:
$$ (k, W_{i,k}, T_{i,k}) \in (0, ..., \sum c_j) \times (0, ..., b) \times Pot({1, ..., n})
    = \text{Nutzen} \times \text{Gewicht} \times \text{Teilmenge} $$
wobei $W_{i,k}$ für Nutzen $k$ minimal ist und $W_{i,k} \leq b$.
Sei $TRIPLE_i$ die Menge alle Tripel für $I_i$.

Iteriere über alle $i$, und alle $TRIPLE_i$, und erweitere die Tripel um das i-te Element.
Gebe das grösste $k$ aus allen gefundenen Tripeln aus.

Laufzeit: $\bigO(|I| \cdot \sum c_j) = \bigO(n \cdot n \cdot \text{max-int}(I)) \implies$ pseudopolynomiell

Falls $b \ll \sum c_j$: speichere Tripel für jedes mögliche Gewicht den maximalen Nutzen
(anstatt für jeden möglichen Nutzen das minimale Gewicht).

\paragraph{FPTAS-KP}
Siehe [ASP].

\begin{enumerate}
    \item $t \gets \frac{\varepsilon \cdot c_{max}}{(1+\varepsilon)\cdot n}$
    \item Runde $c_i' \gets \lfloor \frac{c_i}{t} \rfloor$
    \item DPKP auf gerundete Instanz
\end{enumerate}
Korrektheit: Lösung bleibt zulässig.
Approximationsgüte: messy, siehe Buch/[ASP].\\
Laufzeit: $\bigO (n + n \cdot \sum c_j') = \bigO (\frac{1}{\varepsilon}n^3)$
$\implies$ poly($n, \varepsilon^{-1}$)

\newpage

\input{3-reduktionen}
\newpage


\part{Online-Algorithmen}

\section{Das Paging-Problem}

\begin{takeaway}
    \item Online-Problem, Online-Algorithmus, kompetitiver Faktor
    \item Paging-Problem
\end{takeaway}

\paragraph{Motivation}
Probleme lösen ohne vollständige Informationen zu haben (die für eine optimale Lösung relevant sind).
Stattdessen werden die Informationen stückweise zur Laufzeit bekannt.

\paragraph{Online-Problem}
Ein \emph{Online-Minimierungsproblem} ist $\Pi = (I, O, cost, \min)$.
Eine Eingabe $I = (x_1, ..., x_n) \in \mathcal{I}$ ist eine Folge von \emph{Anfragen}.
Eine akzeptierte Lösung $O = (y_1, ..., y_n)$ ist eine Folge von \emph{Antworten}.

Beim analogen Maximierungsproblem spricht man statt von $cost(I, O)$ oft vom \emph{Gewinn} $gain(I,O)$.

\paragraph{Online-Algorithmus}
Sei $\Pi$ ein Online-Optimierungsproblem.
Ein \emph{Online-Algorithmus} $\A$ berechnet die Ausgabe $\A(I) = (y_1, ..., y_n) $
wobei $y_i$ nur von $(x_1, ..., x_i)$ abhängt.
$\A(I)$ ist eine zulässig Lösung für $I$.

\paragraph{Kompetitive Faktor}
(aka. competitive ratio, Wettbewerbsgüte, kompetitive Güte) \\
Ein Online-Algorithmus $\A$ ist \emph{c-kompetitiv} falls gilt:
\begin{align*}
\exists \alpha \geq 0 \quad \forall I \cl \quad cost(\A(I)) & \leq c \cdot cost(Opt(I)) + \alpha \\
\dfrac{cost(\A(I))}{cost(Opt(I))} + \alpha' & \leq c
\end{align*}
für ein Minimierungsproblem und $\alpha$ konstant.
$Opt$ ist ein optimaler Offline-Algorithmus, d.h. mit vollständiger Information.

Das kleinste $c$ für das dies gilt heisst \emph{kompetitiver Faktor}. \\
Für $\alpha = 0$ heisst $\A$ \emph{strikt-c-kompetitiv}. \\
Falls $\A$ strikt-1-kompetitiv ist ($\alpha = 0, c = 1$) so heisst er \emph{optimal}.

Ein Online-Algorithmus heisst \emph{kompetitiv} wenn sein kompetititver Faktor nicht von der
Länge der Eingabe abhängt.
Wir sprechen dabei von \emph{kompetitiver Analyse}.
Der kompetitiver Faktor ist vergleichbar mit der Approximationsgüte von Approximationsalgorithmen.

Die Konstante $\alpha$ ist wichtig da sie erlaubt auf kurze Eingaben schlecht zu sein
(und erst auf lange besser zu werden).
\footnote{Warum brauchen wir bei der Approximationsgüte keine vergleichbare Konstante?}

\paragraph{Paging}
\begin{itemize}
    \item Eingabe: $ I = (x_1, ..., x_n)$ mit Speicher-Indizes $x_i \in \N$
    \item Hauptspeicher mit $m$ Seiten: $ (s_1, ..., s_m) $
    \item Cache-Speicher mit $k$ Seiten: $ B = (s_{j_1}, ..., s_{j_k}) $, initialisiert mit $ (s_1, ..., s_k) $
        \footnote{Der Vorsprung eines selbstgewählten Startinhalts kann in $\alpha$ versteckt werden.}
    \item Zeitschritt $i$:
    \begin{itemize}
        \item Index $x_i$ wird angefragt
        \item Falls $x_i$ im Cache (d.h. $s_{x_i} \in B$): return $y_i=0$
        \item Andernfalls: return $y_i=j$, und setze $B = B \backslash  \{s_j\} \cup \{s_{x_i}\} $,
            d.h. lösche Seite $s_j$ aus dem Cache.
            \footnote{Zusätzliches, proaktives Entfernen bringt keinen Vorteil.}
    \end{itemize}
    \item $ cost(\A(I)) := \vert \{ i \st y_i > 0 \} \vert $
    \item goal := min
\end{itemize}

Strategien bei \emph{Seitenfehlern (page faults)} zum \emph{Verdrängen} von Seiten:
First-in-First-Out (FIFO, wie eine Queue),
Last-in-First-Out (LIFO, wie ein Stack),
Least-Recently-Used (LRU),
Longest-Forward-Distance (LFD, offline-only!).

\paragraph{Satz}
Ein Online-Algorithmus für Paging der FIFO nutzt ist strikt-k-kompetitiv.

\underline{Beweis:}
Gruppiere Zeitschritte in \emph{Phasen}.
Phase 1 endet nach dem ersten Seitenfehler.
Phase $P \geq 2$ endet nach $1+ (P-1)k$ Seitenfehlern, d.h. alle $k$ Fehler endet eine Phase und beginnt eine neue.

In Phase 1 machen $Opt$ und $Fifo$ je genau einen Fehler (warum?).

Sei $s$ die Seite die den letzten Seitenfehler von Phase $P-1$ verursacht
(d.h. sie kommt neu in den Cache, und wird dank FIFO als letztes in Phase $P$ verdrängt werden). \\
$\implies$ Zu Beginn von Phase $P$ ist $s$ im Cache von $Opt$ \underline{und} von $Fifo$. \\
$\implies$ Es gibt $\leq k-1$ Seiten die im Cache von $Opt$ sind, aber nicht in dem von $Fifo$. \\
Während Phase $P$ macht $Fifo$ genau $k$ Fehler. \\
$\implies$  Während $P$ muss $Opt$ mindestens einen Seitenfehler machen. \\
$\implies$  $Fifo$ ist k-kompetitiv.

LRU ist in der Theorie ebenfalls k-kompetitiv, in der Praxis allerdings tendenziell besser als FIFO.



\newpage


\end{document}
