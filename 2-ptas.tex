\section{Approximationsschemata}

\begin{takeaway}
    \item PTAS, FPTAS
    \item SKP: 2-Approximation (Greedy), PTAS-SKP (Durchschnittsargument)
    \item KP: DPKP, FPTAS
\end{takeaway}

\paragraph{Definition PTAS und FPTAS}
Eingabe $(I, \varepsilon)$.
PTAS: Laufzeit ist polynomiell in $|I|$ und beliebig in $\varepsilon^{-1}$.
FPTAS: Laufzeit ist polynomiell in $|I|$ \underline{und} in $\varepsilon^{-1}$.
Approximationsgüte $(1+\varepsilon)$.
Siehe [ASP].

\paragraph{Einfaches Rucksackproblem (Simple Knapsack Problem SKP)}
Gewichte = Kosten. NP-schwer.
Siehe [ASP].

\paragraph{Greedy-SKP}
2-Approximation. Absteigend sortieren, $\bigO(n \log n)$.
Siehe [ASP].

\paragraph{PTAS-SKP}
$(1+\varepsilon)$-Approximation.
$k \gets \lceil \frac{1}{\varepsilon} \rceil$.
Optimale Lösung für alle $\bigO(n^k)$ Teilmengen der Grösse $k$. Dann greedy erweitern in je $\bigO(n)$.
Kosten mit \emph{Durchschnittsargument}: $w_q \leq \frac{w_1 + ... + w_k + w_q}{k+1} \leq \frac{cost(T_{Opt})}{k+1}$.
Siehe [ASP].

\paragraph{Allgemeines Rucksackproblem (KP)}
Eingabe $I = (w_1, ..., w_n, c_1, ..., c_n, b)$.
Siehe [ASP].

\paragraph{Exakter Algorithmus für KP (DPKP)}
Dynamische Programmierung. Siehe [ASP].

Sei $I_i$ die Teilinstanz der ersten $i$ Elemente.
Berechne Tripel:
$$ (k, W_{i,k}, T_{i,k}) \in (0, ..., \sum c_j) \times (0, ..., b) \times Pot({1, ..., n})
    = \text{Nutzen} \times \text{Gewicht} \times \text{Teilmenge} $$
wobei $W_{i,k}$ für Nutzen $k$ minimal ist und $W_{i,k} \leq b$.
Sei $TRIPLE_i$ die Menge alle Tripel für $I_i$.

Iteriere über alle $i$, und alle $TRIPLE_i$, und erweitere die Tripel um das i-te Element.
Gebe das grösste $k$ aus allen gefundenen Tripeln aus.

Laufzeit: $\bigO(|I| \cdot \sum c_j) = \bigO(n \cdot n \cdot \text{max-int}(I)) \implies$ pseudopolynomiell

Falls $b \ll \sum c_j$: speichere Tripel für jedes mögliche Gewicht den maximalen Nutzen
(anstatt für jeden möglichen Nutzen das minimale Gewicht).

\paragraph{FPTAS-KP}
Siehe [ASP].

\begin{enumerate}
    \item $t \gets \frac{\varepsilon \cdot c_{max}}{(1+\varepsilon)\cdot n}$
    \item Runde $c_i' \gets \lfloor \frac{c_i}{t} \rfloor$
    \item DPKP auf gerundete Instanz
\end{enumerate}
Korrektheit: Lösung bleibt zulässig.
Approximationsgüte: messy, siehe Buch/[ASP].\\
Laufzeit: $\bigO (n + n \cdot \sum c_j') = \bigO (\frac{1}{\varepsilon}n^3)$
$\implies$ poly($n, \varepsilon^{-1}$)
