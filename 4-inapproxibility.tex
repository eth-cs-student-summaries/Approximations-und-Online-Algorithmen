\section{Nichtapproximimierbarkeit}

\begin{takeaway}
    \item Pseudopolynomielle Algorithmen, Stark NP-schwer
    \item AP-Reduktionen, Max-SAT $\leq_{AP}$ Max-CLIQUE
    \item GP-Reduktionen, Lückenproblem
    \item MAX-E3SAT $\leq_{GP}$ MAX-2SAT, Max-CLIQUE $\leq_{GP}$ Max-CLIQUE
    \item Probabilistische Verifizierer, GAP$_{1-\varepsilon, 1}$(E3SAT)
    \item PCP-Theorem
\end{takeaway}

\paragraph{Nichtapproximimierbarkeit}
Motivation: zeige untere Schranken für die Polynomzeit-Approximier-barkeit von Problemen.
Methoden:
\begin{itemize}
    \item Reduktion auf NP-schwere Entscheidungsprobleme
    \item Approximationserhaltende Reduktionen (AP-Reduktionen)
    \item Anwendung PCP-Theorem
\end{itemize}


\subsection{Reduktion auf NP-schwere Entscheidungsprobleme}

\paragraph{Theorem}
Falls $P \neq NP$, so existiert kein polynomieller Approximationsalgorithmus für das TSP
mit Approximationsgüte $p(n)$ für ein Polynom $p$.
$\implies$ TSP $\notin$ POLYAPX.

\underline{Beweis:}
Siehe [ASP]. Reduktion vom Hamiltonkreisproblem HCP (NP-schwer) auf die $2^n$-Approximation von TSP.
Polynomzeit-Transformation.
$$ HCP = \{ G=(V,E) \st G \text{ enthält einen Hamiltonkreis} \} $$
Hamiltonkreis: jeder Knoten wird genau einmal besucht.

\paragraph{Zahlproblem (integer value problem IVP)}
Siehe [ASP].

\paragraph{Pseudopolynomieller Algorithmen}
$\text{time}_A(x) \in \bigO \left( \text{poly}( |I|, \text{max-int}(I) ) \right)$.
Siehe [ASP].

\paragraph{Stark NP-schwer}
Zahlproblem U heisst \emph{stark NP-schwer} falls das $p$-beschränkte Teilproblem%
\footnote{D.h. $\text{max-int}(I) \leq p(|I|)$.} NP-schwer ist, für ein Polynom $p$.
Siehe [ASP].

\paragraph{Theorem}
U stark NP-schwer $\implies$ $\not \exists$ pseudopolynomieller Algorithmus für U\\
Siehe [ASP].
